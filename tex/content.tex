
\section*{Введение}
\addcontentsline{toc}{section}{Введение}
\thispagestyle{plain}

В современном мире наблюдается стремительное развитие программных решений, представленных на рынке баз данных.

\newpage

\section{Текст главы 1}
\thispagestyle{plain}
Соотношение сложности представления обрабатываемых данных и алгоритма вычислений определяет два крайних случая выполняемых с помощью компьютера задач:
\begin{itemize}[noitemsep,nolistsep]
    \item \textit{текст 1} текст 2;
    \item \textbf{текст 1} текст 2.
\end{itemize}


Согласно~\cite{gost33707}, \textit{база данных} (БД) \textemdash~совокупность данных, хранимых в соответствии со схемой данных, манипулирование которыми выполняют в соответствии с правилами средств моделирования данных.

\subsection{Подраздел}

Некоторый текст  Некоторый текст Некоторый текст  Некоторый текст Некоторый текст  Некоторый текст Некоторый текст  Некоторый текст Некоторый текст  Некоторый текст Некоторый текст  Некоторый текст Некоторый текст  Некоторый текст Некоторый текст  Некоторый текст 

Некоторый текст  Некоторый текст Некоторый текст  Некоторый текст Некоторый текст  Некоторый текст Некоторый текст  Некоторый текст Некоторый текст  Некоторый текст Некоторый текст  Некоторый текст Некоторый текст  Некоторый текст 
Некоторый текст  Некоторый текст 

Пример нумерованного списка:
\begin{enumerate}[label=\arabic*., noitemsep,nolistsep]
    \item 123123;
    \item 23141234;
    \item 241243;
\end{enumerate}

Пример вставки кода:
\begin{lstlisting}
# Driver code to test above
if __name__ == "__main__":
    arr = [64, 34, 25, 12, 22, 11, 90]

    bubbleSort(arr)

    print("Sorted array:")
    for i in range(len(arr)):
        print("%d" % arr[i], end=" ")
\end{lstlisting}

Пример вставки картинки на Рис. \ref{fig:imageCat}.

\image{Images/cat.png}{Первая картинка}{fig:imageCat}{0.5}

Пример вставки другой картинки на Рис. \ref{fig:imageDog}.

\image{Images/dog.png}{Вторая картинка}{fig:imageDog}{0.6}

Некоторый текст  Некоторый текст Некоторый текст  Некоторый текст Некоторый текст  Некоторый текст Некоторый текст  Некоторый текст Некоторый текст  Некоторый текст Некоторый текст  Некоторый текст Некоторый текст  Некоторый текст 
Некоторый текст  Некоторый текст \cite{codd}


\section{Текст главы 2}

Некоторый текст  Некоторый текст Некоторый текст  Некоторый текст Некоторый текст  Некоторый текст Некоторый текст  Некоторый текст Некоторый текст  Некоторый текст Некоторый текст  Некоторый текст Некоторый текст  Некоторый текст 
Некоторый текст  Некоторый текст \cite{statistadata}

\begin{table}[!htb]
\fontsize{11pt}{13pt}\selectfont
\centering
\smallskip
\begin{tabularx}{\linewidth}{|c|X|X|}
    \hline
    \multirow{1}{*}{ \textbf{Команда} } & \hfil \multirow{1}{*}{ \textbf{Описание} }  & \hfil \multirow{1}{*}{ \textbf{Пример} } \\
    \hline
        SET & \textit{установка значения ключа} & SET somekey "my string" \\
    \hline
        MSETNX & \textit{установка значений для нескольких ключей} & MSETNX somekey "my string" test:anykey "my string" \\
    \hline
        GET & \textit{получение значения для ключа} & GET somekey \\
    \hline
        MGET & \textit{получение значений нескольких ключей} & MGET somekey1 somekey2 \\
    \hline
        GETSET & \textit{получение и обновление значения ключа } & GETSET somekey "value" \\
    \hline
       RENAME & \textit{переименование ключа } & RENAME somekey anotherkey \\
    \hline
        TYPE & \textit{вывод типа данных для ключа} & TYPE somekey \\
    \hline
        KEYS & \textit{получение всех ключей по маске (*)} & KEYS somekey* \\
    \hline
        DELKEY & \textit{удаление значения для ключа} & DEL somekey \\
    \hline
        EXISTS & \textit{проверка существования значения по ключу} & EXISTS somekey \\
    \hline
        EXPIRE & \textit{удаление ключа по прошествии некоторого времени (в секундах)} & EXPIRE somekey 15 \\
    \hline
        TTL & \textit{вывод числа секунд до удаления значения по ключу } & TTL somekey \\
    \hline
\end{tabularx}
\label{table:redis-base-commands}
\end{table}

\newpage

\section*{Заключение}
\addcontentsline{toc}{section}{Заключение}

Некоторый текст  Некоторый текст Некоторый текст  Некоторый текст Некоторый текст  Некоторый текст Некоторый текст  Некоторый текст Некоторый текст  Некоторый текст 